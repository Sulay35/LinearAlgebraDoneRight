\documentclass[12pt]{book}
\usepackage{stylepackage}

\begin{document}
\section*{Exercices}

\begin{enumerate}
    \item Supposez que $a$ et $b$ sont des nombres réels, non tous nuls. Trouver les nombres réels $c$ et $d$ tels que
    \begin{equation*}
        1/(a+ib)=c+id
    \end{equation*}
    \item Montrez que
    \begin{equation*}
        z=\frac{-1+i\sqrt{3}}{2}
    \end{equation*}
    est une racine cubique de 1 (ce qui veut dire que son cube vaut 1).
    \item Prouvez que si $a\in\F,~v\in E$, et que $av=0$, alors $a=0$ ou $v=0$.
    \item Prouvez que $-(-v)=v$ pour tout $v\in E$
    \item Pour chacun des sous-ensembles de $\F^3$ suivants, déterminez s'il s'agit d'un sous-espace vectoriel de $\F^3$ ou non :
    \begin{description}
        \item[(a)] $ \{(x_1,x_2,x_3)\in\F^3:x_1+2x_2+3x_3=0\} $
        \item[(b)] $ \{(x_1,x_2,x_3)\in\F^3:x_1+2x_2+3x_3=4\} $
        \item[(c)] $ \{(x_1,x_2,x_3)\in\F^3:x_1x_2x_3=0\} $
        \item[(d)] $ \{(x_1,x_2,x_3)\in\F^3:x_1=5x_3\} $
    \end{description}
    \item Prouvez que l'intersection de n'importe quel ensemble de sous-espace vectoriel de $E$ est un sous-espace vectoriel de $E$.
    \item Donnez un exemple d'un sous-ensemble $F$ de $\R^2$ non vide tel que $F$ soit stable pour l’addition et ait un opposé (c'est-à-dire que $-u\in F$ si $u\in F$), mais que $F$ n'est pas un sous-espace vectoriel de $\R^2$.
    \item Donnez un exemple d'un sous-ensemble $F$ de $\R^2$ non vide tel que $F$ soit stable pour la multiplication scalaire, mais que $F$ n'est pas un sous-espace vectoriel de $\R^2$.
    \item Prouvez que l'union de 2 sous-espaces vectoriels de $E$ est un sous-espace vectoriel de $E$ si seulement l'un des sous-espaces vectoriels est inclus dans l'autre.
    \item Supposez que $F$ est un sous-espace vectoriel de $E$. Qu'est-ce que $F+F$ ?
    \item Est-ce que l'opération d'addition sur les sous-espaces vectoriels de $F$ possède un élément neutre ? Quels sous-espaces vectoriels ont un opposé ?
    \item Prouvez ou donnez un contre-exemple : si $F_1, F_2,G$ sont des sous-espaces vectoriels de $E$ tels que
    \begin{equation*}
        F_1+G=F_2+G
    \end{equation*}
    alors $F_1=F_2$.
    \item Est-ce que l'addition sur des sous-espaces vectoriels de $E$ est commutative ? Associative ? (Autrement dit, si $F_1,F_2,F_3$ sont des sous-espaces vectoriels de $E$, est-ce que $F_1+F_2=F_2+F_1$ ? Est-ce que $(F_1+F_2)+F_3=F_1+(F_2+F_3)$ ?)
    \item Supposez que $F$ est le sous-espace vectoriel de $\mathcal{P}(\F)$ constitué de tous les polynômes $p$ de la forme
    \begin{equation*}
        p(z)=az^2+bz^5
    \end{equation*}
    où $a,b\in\F$. Trouvez un sous-espace vectoriel $G$ de $\mathcal{P}(\F)$ tel que $\mathcal{P}(\F)=F\oplus G$.
    \item Prouvez ou donnez un contre-exemple : si $F_1,F_2,G$ sont des sous-espaces vectoriels de $E$ tels que
    \begin{equation*}
        E=F_1\oplus G \mathrm{~et~} E=F_2\oplus G
    \end{equation*}
    alors $F_1=F_2$.
\end{enumerate}

















\end{document}