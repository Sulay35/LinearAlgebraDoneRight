\documentclass[12pt]{book}
\usepackage{stylepackage}

\begin{document}
\section*{Exercices}
\begin{enumerate}
\item Prouvez que si $(v_1,\ldots,v_m)$ engendre $V$ , il en va de même pour la famille
\begin{equation*}
 (v_1-v_2,v_2-v_3,\ldots,v_n{_-}_1-v_n,v_n)   
\end{equation*}
obtenu en soustrayant de chaque vecteur (sauf le dernier) le vecteur suivant.
\item Prouvez que si $(v_1,\ldots,v_m)$ est lin\'eairement ind\'ependant dans $V$, alors la famille l’est aussi
\begin{equation*}
 (v_1-v_2,v_2-v_3,\ldots,v_n{_-}_1-v_n,v_n)   
\end{equation*}
obtenu en soustrayant de chaque vecteur (sauf le dernier) le vecteur suivant.
\item Supposons que $(v_1,\ldots,v_m)$ est lin\'eairement ind\'ependant dans $V$ et $w$ $\in$ $V$.\\
Prouvez que si $(v_1 + w,\ldots,v_n +w)$ est lin\'eairement ind\'ependant, alors $w$ $\in$ $span(v_1,\ldots,v_m)$.
\item Supposons que $m$ soit un nombre entier positif. L'ensemble constitu\'e de $0$ et de tous les polynômes à coefficients dans $\F$ et de degr\'e \'egal \`a $m$ est-il un sous-espace de $P(\F)$ ?
\item

\item Prouvez que l’espace vectoriel r\'eel constitu\'e de toutes les fonctions continues \`a valeur r\'eelle sur l’intervalle [0,1] est de dimension infinie.
\item Prouvez que V est de dimension infinie si et seulement s’il existe une suite v1,v2,ldots de vecteurs dans V telle que (v1,ldots,vn) est linéairement indépendante pour tout entier positif $n$.




















\end{enumerate}

\end{document}