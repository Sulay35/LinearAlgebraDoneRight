\documentclass[12pt]{book}
\usepackage{stylepackage}

\begin{document}
\section*{Correction du chapitre 1}

\subsection*{Exercice 1}


\subsection*{Exercice 3}


\subsection*{Exercice 4}


\subsection*{Exercice 5}


\subsection*{Exercice 6}


\subsection*{Exercice 7}


\subsection*{Exercice 8}


\subsection*{Exercice 9}

    
\subsection*{Exercice 10}


\subsection*{Exercice 11}


\subsection*{Exercice 12}


\subsection*{Exercice 13}


\subsection*{Exercice 14}


\subsection*{Exercice 15}


\subsection*{Exercice 16}


\subsection*{Exercice 17}
\textit{Prouvons que la multiplication matricielle et l'addition matricielle sont reliées par la distributivité, autrement dit que $A(B+C)=AB+AC$.} 
Pour que $A(B+C)$ fasse sens, il faut d'abord que $B+C$ soit correct, alors $B$ et $C$ doivent être de la même taille, ensuite le produit matriciel doit respecter la règle sur les tailles. Alors on peut définir $A\in\M_{m,n}$ et $B,C\in\M_{n,p}$.
Ainsi 
\begin{equation*}
    A = 
    \begin{bmatrix}
    a_{1,1} & \ldots & a_{1,n} \\
    \vdots & & \vdots \\
    a_{m,1} & \ldots & a_{m,n}
    \end{bmatrix}
    B = 
    \begin{bmatrix}
    b_{1,1} & \ldots & b_{1,p} \\
    \vdots & & \vdots \\
    b_{n,1} & \ldots & b_{n,p}
    \end{bmatrix}
    C = 
    \begin{bmatrix}
    c_{1,1} & \ldots & c_{1,p} \\
    \vdots & & \vdots \\
    c_{n,1} & \ldots & c_{n,p}
    \end{bmatrix}
\end{equation*}

\begin{equation*}
    B+C = 
    \begin{bmatrix}
    b_{1,1} + c_{1,1} & \ldots & b_{1,p} + c_{1,p} \\
    \vdots & & \vdots \\
    b_{n,1} + c_{n,1} & \ldots & b_{n,p}+c_{n,p}
    \end{bmatrix}
\end{equation*}

Alors 
\begin{equation*}
\begin{split}
    A(B+C)& =
    \begin{bmatrix}
    a_{1,1} & \ldots & a_{1,n} \\
    \vdots & & \vdots \\
    a_{m,1} & \ldots & a_{m,n}
    \end{bmatrix}
    \begin{bmatrix}
    b_{1,1} + c_{1,1} & \ldots & b_{1,p} + c_{1,p} \\
    \vdots & & \vdots \\
    b_{n,1} + c_{n,1} & \ldots & b_{n,p}+c_{n,p}
    \end{bmatrix}\\
    & = 
    \begin{bmatrix}
    \sum\limits\limits_{j=1}^na_{1,j}(b_{j,1} + c_{j,1}) & \ldots & \sum\limits_{j=1}^na_{1,j}(b_{j,p} + c_{j,p}) \\
    \vdots & & \vdots \\
    \sum\limits_{j=1}^na_{m,j}(b_{j,1} + c_{j,1}) & \ldots & \sum\limits_{j=1}^na_{m,j}(b_{j,p} + c_{j,p})
    \end{bmatrix}\\
    & =
    \begin{bmatrix}
    \sum\limits_{j=1}^na_{1,j}b_{j,1}+\sum\limits_{j=1}^na_{1,j}c_{j,1}& \ldots & \sum\limits_{j=1}^na_{1,j}b_{j,p}+\sum\limits_{j=1}^na_{1,j}c_{j,p}  \\
    \vdots & & \vdots \\
    \sum\limits_{j=1}^na_{m,j}b_{j,1} + \sum\limits_{j=1}^na_{m,j}c_{j,1} & \ldots & \sum\limits_{j=1}^na_{m,j}b_{j,p}+ \sum\limits_{j=1}^na_{m,j}c_{j,p}
    \end{bmatrix}\\
\end{split}
\end{equation*}

Or 
\begin{equation*}
\begin{split}
 AB+AC& =
    \begin{bmatrix}
    \sum\limits_{j=1}^na_{1,j}b_{j,1} & \ldots & \sum\limits_{j=1}^na_{1,j}b_{j,p}  \\
    \vdots & & \vdots \\
    \sum\limits_{j=1}^na_{m,j}b_{j,1}  & \ldots & \sum\limits_{j=1}^na_{m,j}b_{j,p}
    \end{bmatrix}+
    \begin{bmatrix}
    \sum\limits_{j=1}^na_{1,j}c_{j,1}& \ldots & \sum\limits_{j=1}^na_{1,j}c_{j,p}  \\
    \vdots & & \vdots \\
    \sum\limits_{j=1}^na_{m,j}c_{j,1} & \ldots & \sum\limits_{j=1}^na_{m,j}c_{j,p}
    \end{bmatrix}\\
    & =
    \begin{bmatrix}
    \sum\limits_{j=1}^na_{1,j}b_{j,1}+\sum\limits_{j=1}^na_{1,j}c_{j,1}& \ldots & \sum\limits_{j=1}^na_{1,j}b_{j,p}+\sum\limits_{j=1}^na_{1,j}c_{j,p}  \\
    \vdots & & \vdots \\
    \sum\limits_{j=1}^na_{m,j}b_{j,1} + \sum\limits_{j=1}^na_{m,j}c_{j,1} & \ldots & \sum\limits_{j=1}^na_{m,j}b_{j,p}+ \sum\limits_{j=1}^na_{m,j}c_{j,p}
    \end{bmatrix}
\end{split}
\end{equation*}

\textbf{Conclusion} 
\begin{equation*}
    A(B+C) = 
    \begin{bmatrix}
    \sum\limits_{j=1}^na_{1,j}b_{j,1}+\sum\limits_{j=1}^na_{1,j}c_{j,1}& \ldots & \sum\limits_{j=1}^na_{1,j}b_{j,p}+\sum\limits_{j=1}^na_{1,j}c_{j,p}  \\
    \vdots & & \vdots \\
    \sum\limits_{j=1}^na_{m,j}b_{j,1} + \sum\limits_{j=1}^na_{m,j}c_{j,1} & \ldots & \sum\limits_{j=1}^na_{m,j}b_{j,p}+ \sum\limits_{j=1}^na_{m,j}c_{j,p}
    \end{bmatrix}
    = AB + AC
\end{equation*}
Nous avons prouvé que la multiplication matricielle et l'addition matricielle sont reliées par la distributivité. \hfill $\blacksquare$

\end{document}