\documentclass[12pt]{book}
\usepackage[french]{babel}

\usepackage[utf8]{inputenc}
\usepackage[T1]{fontenc}

\usepackage{amsmath}
\usepackage{amssymb}
\usepackage{eucal}

\usepackage{graphicx,graphics}
\graphicspath{ {images/} } % les images utilisées sont dans le dossier image

\usepackage{epstopdf}
\usepackage{amsthm}

\numberwithin{equation}{section}

\usepackage{fancyhdr}
\pagestyle{fancy}

%%%%%%%%%%%%%%%%%%%%%%%%%%%%%%%%%%%%
%%%%%%%%%%%%%%%%%%%%%%%%%%%%%%%%%%%%

\newtheorem{thm}{Théorème}%[section]
\newtheorem{defn}{Définition}%[section]
\newtheorem{prop}{Proposition}%[section]
\newtheorem{propri}{Propriété}%[section]
\newtheorem{rem}{Remarque}%[section]
\newtheorem{coro}{Corollaire}%[section]
\newtheorem{exemp}{Exemple}%[section]

\newcommand{\N}{\mathbb{N}}
\newcommand{\Z}{\mathbb{Z}}
\newcommand{\Q}{\mathbb{Q}}
\newcommand{\R}{\mathbb{R}}
\newcommand{\C}{\mathbb{C}}
\newcommand{\K}{\mathbb{K}}
\newcommand{\M}{\mathbb{M}}

\def \al {\alpha}
\def \be {\beta}
\def \ga {\gamma}
\def \ep {\varepsilon}


\fancyhead[R]{\leftmark}

\title{L'al}

\begin{document}


%            %
% CHAPITRE 1 %
%            %

\thispagestyle{plain}
\chapter*{\textmd{\textsc{\Large{Chapitre 1}}} \\ \textsl{Espaces vectoriels}}

\vfill


% INTRODUCTION
\indent{L'alg\`ebre lin\'eaire est l'\'etude des applications lin\'eaires sur des espaces vectoriels de dimensions finies. Nous apprendrons plus tard ce que signifient ces termes. Dans ce chapitre, nous allons d\'efinir les espaces vectoriels et aborder leurs propri\'et\'es \'el\'ementaires.}

\indent{Dans certaines parties des math\'ematiques, y compris l'alg\`ebre lin\'eaire, on obtient de meilleurs th\'eor\`emes et une meilleure compr\'ehension du sujet si l'on prend en compte les nombres complexes comme les nombres r\'eels. C'est pourquoi nous allons commencer par introduire les nombres complexes et leurs propri\'et\'es les plus basiques.}

\vfill

\begin{center}
    \includegraphics[scale=0.15]{Asterisk.png}
\end{center}

    

\pagebreak


\end{document}
